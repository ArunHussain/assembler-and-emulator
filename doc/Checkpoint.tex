\documentclass[11pt]{article}

\usepackage{fullpage}
\usepackage{enumitem}

\begin{document}

\title{ARM Checkpoint}
\author{Arun Hussain, Bolei Chen, Haotian (Steven) Qi, Bingdong (Owen) Li}

\maketitle

\section{Group Organisation}
We divided the emulator into the following tasks

\begin{enumerate}
    \item Reading a binary file instruction by instruction.
    \item Representing and manipulating memory and registers.
    \item Decoding and executing data processing (immediate) instructions.
    \item Decoding and executing data processing (register) instructions.
    \item Decoding and executing single data transfer instructions.
    \item Decoding and executing branch instructions.
    \item Carrying out the fetch-decode-execute cycle.
\end{enumerate}

We estimated the workload for each task and attempted to distribute them evenly
beforehand, but left the simpler tasks (items 1., 2., and 6.) unallocated, so we
could rebalance the workload in case our initial estimates were incorrect.

\begin{description} [align=right,labelwidth=3cm]
    \item [Arun Hussain] Decoding and executing branch instructions.
    \item [Bolei Chen] Decoding and executing data processing (immediate) instructions.
    \item [Haotian Qi] Decoding and executing data processing (register) instructions.
    \item [Bingdong Li] Decoding and executing single data transfer instructions.
\end{description}

In practice, single data transfer instructions and branch instructions were easier to
implement. Due to Mr. Hussain being occupied by a then-upcoming interview on 2023-6-6,
Mr. Li implemented the three simpler tasks (1., 2., and 6.), as well as implementing
some utility functions as described in the following section.

\section{Implementation Strategies}

We implemented the \verb|decode_segments| function, which decoded a binary instruction
and stored a variable number of binary segments of specifiable lengths in provided pointers.
This function simplified the decoding of data processing instructions and single data transfer
instructions (branch instructions had a simple enough structure that ad hoc decoding was more
efficient).

Instead of decoding object code into an internal representation, and then decoding that,
we decided instead to merge the decoding and execution steps. The highest level function,
\verb|decode_execute|, would decide the type of each instruction and delegate decoding and
execution to instruction type specific procedures. This modular structure allowed us to
cooperate without creating frequent merge conflicts. Due to their simplicity, this function
also executed the special instructions (nop and halt), and also incremented the program
counter for non-branch instructions.

Each instruction type specific procedure would similarly decide the subtype, decode the
shared portions (e.g., \verb|rd| for data processing (immediate) instructions),
and pass them (as well as the undecoded parts) as arguments to further procedures that 
actually executes the instructions, doing any further decoding as necessary. This meant
that only one procedure would be responsible for the decoding and execution of each
instruction subtype, and hence debugging could be concentrated on a small segment of the code.
We reasoned that the subtype deciding code would be easy to debug even in the event of
a mistake, simply by inspecting standard debugging output. In practice, no such mistake
was made.

Instead of allowing ad hoc setting of registers and memory locations, we defined standard
functions and procedures for reading and manipulating memory and the register file (task 1.).
This allowed some conditions (e.g., the zero register should ignore any sets) to be easily
implemented. We stored memory as an array of \verb|uint_8|s (i.e., bytes), and registers
as an array of of 34 \verb|uint_64|s (at the time, it was not clear if we could ignore
everything stack pointer related in the spec). We placed the zero register at position 31,
so that the common practice of treating a reference to register \verb|0b1111| as a reference
to the zero register would require minimal additional code.

Overall, we adopted a strategy of centrally defining shared code for common, repetitive tasks,
using procedure calls to implicity traverse the decision tree, and allocating responsibility
by function so that implementation details for the decoding and execution of each individual
instruction can be left to a single person, thus minimizing implementation conflicts.

\end{document}
